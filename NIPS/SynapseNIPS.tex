\documentclass{article} % For LaTeX2e
\usepackage{nips13submit_e,times}
\input{sl_preamble.tex}
\input{sl_graphics_preamble.tex}
\graphicspath{{"Figs/"}}
% >> Only for drafts! <<
\usepackage[notref,notcite]{showkeys}
% ----------------------------------------------------------------
%\numberwithin{equation}{section}
%\renewcommand{\baselinestretch}{1.5}
% ----------------------------------------------------------------
% New commands etc.
\input{sl_definitions.tex}
\input{sl_symbols.tex}
%
%additional symbols:
%
\DeclareMathOperator{\SNR}{SNR}
%matrices
\newcommand{\inv}{^{-1}}
\newcommand{\dg}{^\mathrm{dg}}
\newcommand{\trans}{^\mathrm{T}}
\newcommand{\I}{\mathbf{I}}
%vec of ones
\newcommand{\onev}{\mathbf{e}}
%mat of ones
\newcommand{\onem}{\mathbf{E}}
%Markov matrix
\newcommand{\MM}{\mathbf{Q}}
%prob distributions
\newcommand{\pr}{\mathbf{p}}
\newcommand{\eq}{\pr^\infty}
%first passage times
\newcommand{\fpt}{\mathbf{T}}
%off-diag first passage times
\newcommand{\fptb}{\overline{\fpt}}
%fundamental matrix
\newcommand{\fund}{\mathbf{Z}}
%other symbols for matrices
\newcommand{\Pb}{\mathbf{P}}
\newcommand{\D}{\mathbf{D}}
\newcommand{\pib}{\boldsymbol{\pi}}
\newcommand{\Lb}{\boldsymbol{\Lambda}}
\newcommand{\w}{\mathbf{w}}
\newcommand{\uv}{\mathbf{u}}
\newcommand{\vv}{\mathbf{v}}
\newcommand{\W}{\mathbf{W}}
\newcommand{\M}{\mathbf{M}}
\newcommand{\enc}{\mathbf{q}}
\newcommand{\frg}{\W^{\mathrm{F}}}
\newcommand{\F}{\boldsymbol{\Phi}}
%superscripts
\newcommand{\pot}{^{\text{pot}}}
\newcommand{\dep}{^{\text{dep}}}
\newcommand{\potdep}{^{\text{pot/dep}}}
%sets
\newcommand{\CS}{\mathcal{S}}
\newcommand{\CA}{\mathcal{A}}
\newcommand{\CB}{\mathcal{B}}
\newcommand{\comp}{^\mathrm{c}}

%
%%%%%%%%%%%%%%%%%%%%%%%%%%%%%%%%%%%%%%%%%%%%%%%%%%%%%%%%%%%%%%%%%%%%%%%%%%
% Title info:
\title{A memory frontier for complex synapses}
%
% Author List:
%
\author{Subhaneil Lahiri and Surya Ganguli\\
Applied Physics Department, Stanford University, Stanford CA\\
\emaillink{sulahiri@stanford.edu}, \emaillink{sulahiri@stanford.edu}
%
}

%\nipsfinalcopy % Uncomment for camera-ready version

\begin{document}

\maketitle


%%%%%%%%%%%%%%%%%%%%%%%%%%%%%%%%%%%%%%%%%%%%%%%%%%%%%%%%%%%%%%%%%%%%%%%%%%


\begin{abstract}
  Blah blah blah.
\end{abstract}


%%%%%%%%%%%%%%%%%%%%%%%%%%%%%%%%%%%%%%%%%%%%%%%%%%%%%%%%%%%%%%%%%%%%%%%%%%

\section{Introduction}\label{sec:intro}





\section{Mathematical setup}\label{sec:setup}

We use a well established formalism for the study of learning and memory with complex synapses (see \cite{Fusi2005cascade,Fusi2007multistate,Barrett2008discrete}).
In this approach, potentiating and depressing plasticity events occur at random times, with all information about the neural activity and learning rules responsible for them absorbed into their rates.

We make the following assumptions:
\begin{itemize}
  \item There are $N$ identical synapses with $M$ internal functional states each.
  \item There are no spatial or temporal correlations in the pattern of potentiating and depressing event.
  \item There are no correlations in the states of different synapse.
  \item Plasticity events occur at random (Poisson distributed) times at rate $r$.
  \item The fraction of these that are potentiating or depressing are given by $f\pot$ and $f\dep$ respectively.
\end{itemize}



\section{Upper bounds}\label{sec:bounds}





\section{Memory curve envelope}\label{sec:env}





\section{Discussion}\label{sec:disc}





%\subsubsection*{Acknowledgements}


%%%%%%%%%%%%%%%%%%%%%%%%%%%%%%%%%%%%%%%%%%%%%%%%%%%%%%%%%%%%%%%%%%%%%%%%%%

\bibliographystyle{utcaps_sl}
\bibliography{maths,neuro}


\end{document}
