% -*- TeX -*- -*- UK -*-
% ----------------------------------------------------------------
% arXiv Paper ************************************************
%
% Subhaneil Lahiri's template
%
% Before submitting:
%    Comment out hyperref
%    Comment out showkeys
%    Replace \input{?.tex} with its contents
%       or include ?.tex in zip/tar file
%    Put this file, the .bbl file, any picture or
%       other additional files and natbib.sty
%       file in a zip/tar file
%
% **** -----------------------------------------------------------
\documentclass[12pt]{article}
%Preamble:
\usepackage{a4wide}
\input{sl_preamble.tex}
%
% >> Only for drafts! <<
\usepackage[notref,notcite]{showkeys}
% ----------------------------------------------------------------
%\numberwithin{equation}{section}
%\renewcommand{\baselinestretch}{1.5}
% ----------------------------------------------------------------
% New commands etc.
\input{sl_definitions.tex}
\input{sl_symbols.tex}
%matrices
\newcommand{\inv}{^{-1}}
\newcommand{\dg}{^\mathrm{dg}}
\newcommand{\trans}{^\mathrm{T}}
\newcommand{\I}{\mathbf{I}}
%vec of ones
\newcommand{\onev}{\mathbf{e}}
%mat of ones
\newcommand{\onem}{\mathbf{E}}
%Markov matrix
\newcommand{\MM}{\mathbf{Q}}
%equilibrium distribution
\newcommand{\eq}{\mathbf{p}^\infty}
%first passage times
\newcommand{\fpt}{\mathbf{T}}
%off-diag first passage times
\newcommand{\fptb}{\overline{\fpt}}
%fundamental matrix
\newcommand{\fund}{\mathbf{Z}}
%other symbols
\newcommand{\Pb}{\mathbf{P}}
\newcommand{\D}{\mathbf{D}}
\newcommand{\pib}{\boldsymbol{\pi}}
\newcommand{\Lb}{\boldsymbol{\Lambda}}
\newcommand{\w}{\mathbf{w}}
\newcommand{\uv}{\mathbf{u}}
\newcommand{\vv}{\mathbf{v}}
\newcommand{\W}{\mathbf{W}}
\newcommand{\M}{\mathbf{M}}
\newcommand{\enc}{\mathbf{q}}
\newcommand{\frg}{\W^{\mathrm{F}}}
\newcommand{\F}{\boldsymbol{\Phi}}
\newcommand{\syn}{\vec{w}}
\newcommand{\synid}{\syn_\text{ideal}}
\DeclareMathOperator{\SNR}{SNR}
\DeclareMathOperator{\snr}{SNR}
\DeclareMathOperator{\env}{Env}
\newcommand{\rh}{\hat{r}}
\newcommand{\CI}{\mathcal{I}}
\newcommand{\CS}{\mathcal{S}}
\newcommand{\CA}{\mathcal{A}}
\newcommand{\CB}{\mathcal{B}}
\newcommand{\comp}{^\mathrm{c}}
% ----------------------------------------------------------------
\input{sl_theorems_preamble.tex}
% ----------------------------------------------------------------
%
%%%%%%%%%%%%%%%%%%%%%%%%%%%%%%%%%%%%%%%%%%%%%%%%%%%%%%%%%%%%%%%%%%%%%%%%%%
% Title info:
\title{Laplacian envelope}
%
% Author List:
%
\author{Subhaneil Lahiri
%
}

\begin{document}

\maketitle


%%%%%%%%%%%%%%%%%%%%%%%%%%%%%%%%%%%%%%%%%%%%%%%%%%%%%%%%%%%%%%%%%%%%%%%%%%


\begin{abstract}
  We try to find the continuous time Markov process that has the maximal Laplace tranformed signal-to-noise curve.
\end{abstract}

\tableofcontents

%%%%%%%%%%%%%%%%%%%%%%%%%%%%%%%%%%%%%%%%%%%%%%%%%%%%%%%%%%%%%%%%%%%%%%%%%%
% Beginning of Article:
%%%%%%%%%%%%%%%%%%%%%%%%%%%%%%%%%%%%%%%%%%%%%%%%%%%%%%%%%%%%%%%%%%%%%%%%%%


\section{Recognition memory}\label{sec:recog}


We will be trying to store patterns in a set od $N$ synaptic weights, $\syn$.
Every time we try to store a pattern, these synapses are subjected to a plasticity event where each synapse is either potentiated or depressed, depending on the pattern.
we will assume that these patterns are spatially and temporally independent.

At some time, suppose we wish to determine if a given pattern is one of those that we previously attempted to store.
We wish to answer this question by looking at the synaptic weights directly (ideal observer).
For that given pattern there will be an ideal set of synaptic weights, $\synid$, where those synapses that were supposed to be potentiated are maximised and those that were supposed to be depressed are minimised.
Suppose that the given pattern was actually seen at time 0 and we are observing the synapses at time $t$.
The actual set of synaptic weights we see, $\syn(t)$, will be a vector of random variables that differs from $\synid$ due to the stochasticity of the pattern encoding and all of the other (uncorrelated) pattern that are stored after it.
As $t\to\infty$, the synaptic weights will become independent of the patter stored at $t=0$.
Thus, the vector of random variables $\syn(\infty)$ also describes the synaptic weights under the null hypothesis -- if that given pattern had never been stored.

We can test if the pattern had been previously stored by computing $\synid\cdt\syn$ and comparing it to some threshold.
For large $N$, this quantity will have a Gausssian distribution.
There will be a ROC curve as a function of this threshold:
%
\begin{equation}\label{eq:ROC}
  \begin{aligned}
  \operatorname{TPR} &= \Phi_c \prn{ \frac{ \Phi_c\inv(\operatorname{FPR}) - \snr }{ \operatorname{NNR} } },
  \quad\text{where } &
    \Phi_c(x) &= \int_x^\infty \frac{ \e^{-\frac{z^2}{2}} }{ \sqrt{2\pi} }\, \dz, \\&&
    \snr &= \frac{ \av{\synid\cdt\syn(t)} - \av{\synid\cdt\syn(\infty)} }{ \sqrt{\var(\synid\cdt\syn(\infty))}}, \\&&
    \operatorname{NNR} &= \sqrt{\frac{ \var(\synid\cdt\syn(t)) }{ \var(\synid\cdt\syn(\infty)) }},
  \end{aligned}
\end{equation}
%
and TPR/FPR are the true/false positive rates.
When the signal-to-noise ratio, $\snr$ is larger than $\Phi_c\inv(\operatorname{FPR})$, it is beneficial to decrease the noise-to-noise ratio, NNR.
In the other case, it is beneficial to increase it.
The expectations and variances are over the probability distribution of the synaptic states given the sequence of plasticity events and the probability distribution of the sequence of plasticity events.

The formulae above assumed that we know the time between storage and recognition.
We should also compute the expectations and variances over probability distribution of the recall time, $t$, as well.
If we only know average time, $\tau$, a natural choice for this distribution is
%
\begin{equation}\label{eq:recogtime}
  P(t\vert\tau) = \frac{\e^{-t/\tau}}{\tau}.
\end{equation}
%
Different parts of the brain, that store memories for different timescales, could be characterised by different values of $\tau$.





%\subsection*{Acknowledgements}



%%%%%%%%%%%%%%%%%%%%%%%%%%%%%%%%%%%%%%%%%%%%%%%%%%%%%%%%%%%%%%%%%%%%%%%%%%
%\subsection*{Appendices}
%\appendix
%%%%%%%%%%%%%%%%%%%%%%%%%%%%%%%%%%%%%%%%%%%%%%%%%%%%%%%%%%%%%%%%%%%%%%%%%%





%%%%%%%%%%%%%%%%%%%%%%%%%%%%%%%%%%%%%%%%%%%%%%%%%%%%%%%%%%%%%%%%%%%%%%%%%%

\bibliographystyle{utcaps_sl}
\bibliography{maths,neuro,markov}

\end{document}
